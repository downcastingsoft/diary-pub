\RequirePackage{plautopatch}
\documentclass{jlreq}

\usepackage{aliascnt}
\usepackage{amsmath,amssymb,amsthm}
\usepackage{anyfontsize}
\usepackage[backend=biber,style=alphabetic,]{biblatex}
\usepackage{bm,bbm}
\usepackage{cancel}
\usepackage{color}
\usepackage{fancyvrb}
\usepackage{fontspec}
\usepackage{mathrsfs}
\usepackage{mathtools}
\usepackage{url}

% "Make sure it comes last of your loaded packages."
\usepackage[luatex,unicode,hidelinks,pdfusetitle]{hyperref}
\usepackage{cleveref}

% biblatex
\addbibresource{refs.bib}

% cleveref
\theoremstyle{definition}
\newtheorem{theorem}{定理}
\newtheorem{corollary}[theorem]{系}
\newtheorem{definition}[theorem]{定義}
\newtheorem{example}[theorem]{例}
\newtheorem{lemma}[theorem]{補題}
\newtheorem{proposition}[theorem]{命題}
\newtheorem{remark}[theorem]{注意}
\newtheorem{conjecture}[theorem]{予想}

\crefname{theorem}{定理}{定理}
\crefname{corollary}{系}{系}
\crefname{definition}{定義}{定義}
\crefname{example}{例}{例}
\crefname{lemma}{補題}{補題}
\crefname{proposition}{命題}{命題}
\crefname{remark}{注意}{注意}
\crefname{conjecture}{予想}{予想}

\crefname{chapter}{第}{第}
\creflabelformat{chapter}{#2#1章#3}
\crefname{section}{第}{第}
\creflabelformat{section}{#2#1節#3}
\crefname{subsection}{第}{第}
\creflabelformat{subsection}{#2#1節#3}

\crefname{figure}{図}{図}
\crefname{table}{表}{表}
\crefname{equation}{}{}
\creflabelformat{equation}{#2(#1)式#3}

% enumerate
\renewcommand{\labelenumi}{(\arabic{enumi})\quad}


\allowdisplaybreaks

\newcommand{\functype}[3]{#1\colon#2\longrightarrow#3}
\newcommand{\mcL}{\mathcal{L}}
\newcommand{\polynoms}[2]{#1[#2]}
\newcommand{\realnums}{\mathbb{R}}
\newcommand{\sequence}[3]{\sequenceparen{#1}{#2}{#3}{(}{)}} % chktex 9
\newcommand{\sequenceparen}[5]{{#4#1#5}_{#2}^{#3}}
\newcommand{\setbrace}[1]{\left\{#1\right\}} % chktex 21
\newcommand{\setbuilder}[2]{\setbuilderparen{#1}{#2}{\left\lbrace}{\right\rbrace}}
\DeclareMathOperator{\spanset}{span}


\begin{document}

\title{ChiharaのTheorem 3.1 (Existence of OPS)\thanks{%
  最終更新日:{\today}.\\
  Document ID: \texttt{d282e42033b88010fe03ed85ee395b22}.
  URL: \url{https://www.downcastingsoft.net/7e5/diary/article/2023/05/20/hankel/}.
  Date: 2023.05.20.
}}
\author{hnagamin}
\date{2022年5月20日}
\maketitle

\begin{definition}
  実線形写像\(\functype{\mcL}{\polynoms{\realnums}{x}}{\realnums}\)を\emph{モーメント汎関数}と呼ぶ.
  このとき,非負整数\(n\)に対して,
  \begin{align}
    \mu_n &= \mcL[x^n]
  \end{align}
  を\(\mcL\)の\(n\)次\emph{モーメント}という.
  列\(\sequence{\mu_n}{n=0}{\infty}\)を\(\mcL\)の\emph{モーメント列}という.

  実多項式の無限列\(\sequence{p_n}{n=0}{\infty}\)が\(\mcL\)に関する\emph{直交多項式系}であるとは,
  各整数\(n\:(\ge0)\)に対して
  次の条件がすべて成り立つことをいう.
  \begin{enumerate}
    \item \(p_n\)は\(n\)次多項式.
    \item \(\mcL[(p_n)^2]\neq0\).
    \item 各整数\(k\:(0\le k<n)\)に対して\(\mcL[p_n p_k]=0\).
    \qed
  \end{enumerate}
\end{definition}

\begin{lemma}%
  \label{thm:orthp.basis}
  \(\functype{\mcL}{\polynoms{\realnums}{x}}{\realnums}\)をモーメント汎関数,
  \(\sequence{p_n\in\polynoms{\realnums}{x}}{n=0}{\infty}\)を\(\mcL\)に関する直交多項式系とする.
  このとき,任意の多項式\(q\in\polynoms{\realnums}{x}\)は\(p_0,p_1,\dots,p_n\:(n=\deg q)\)の線形結合で書け,
  しかもその係数は一意的である.
\end{lemma}
\begin{proof}
  \(n\)に関する帰納法で示す.

  \(n=0\)のとき,
  \(\mcL[p_0^2]\neq0\)に注意して\(v_0=q/p_0\)とおくと\(q=v_0 p_0\)である.

  \(n>0\)とする.
  \(q\)の最高次の係数を\(Q\:(\neq0)\),\(p_n\)の最高次の係数を\(a_{nn}\:(\neq0)\)とする.
  \(v_n=Q/a_{nn}\)とおくと,\(q-v_n p_n\)は\(n-1\)次多項式だから,
  帰納法の仮定より\(v_0,\dots,v_{n-1}\in\realnums\)が存在して
  \begin{align}
    q-v_n p_n &= \sum_{i=0}^{n-1} v_i p_i
  \end{align}
  が成り立つ.よって\(q\in\spanset{\setbrace{p_0,\dots,p_n}}\)である.

  上の議論から,任意の\(n\)次多項式\(q\)は\(q=v_0 p_0+\cdots+v_n p_n\)という形で書けることが分かった.
  いま,\(i\)を\(n\)以下の非負整数とすると,
  \begin{align}
    \mcL[p_i q]
    &= \sum_{j=0}^n v_j \mcL[p_i p_j]
     = v_i \mcL[p_i^2]
    \label{eq:orthp.basis.piq}
  \end{align}
  だから\(v_i=\mcL[p_i q]/\mcL[p_i^2]\).
  よって各\(v_i\)は一意的に定まる.
\end{proof}

いま非負整数\(n\)を任意に固定し,
\(q=u_0+u_1x+\cdots+u_n x^n\)を高々\(n\)次の多項式とする.
\cref{thm:orthp.basis}より,\(q\)は以下のような形で一意的に書ける.
\begin{align}
  q &= v_0 p_0+v_1 p_1+\cdots+v_n p_n
\end{align}
上式から\((u_0,\dots,u_n)\longmapsto(v_0,\dots,v_n)\)という対応が定められるが,
これは可逆かつ線形だから線形同型である.
この同型は見るからに重要なので,具体的に中身を書き下してみる.
\(K_n=\mcL[p_n^2]\)とおくと,
\cref{eq:orthp.basis.piq}より
\begin{align}
  v_i
  &= \frac{\mcL[p_i q]}{K_i}
  = \sum_{i=0}^n \frac{\mcL[p_i x^j]}{K_i} u_j
  & &(i=0,1,\dots,n)
\end{align}
である.
\(T_{ij}=\mcL[p_i x^j]/K_i\)とおくと,行列\(T=(T_{ij})_{ij}\)は上三角であって
\begin{align}
  \begin{pmatrix}
    v_0 \\ \vdots \\ v_n
  \end{pmatrix}
  &=
  \begin{pmatrix}
    T_{00} & \cdots & T_{0n} \\
    \vdots & \ddots & \vdots \\
    0 & \cdots & T_{nn}
  \end{pmatrix}
  \begin{pmatrix}
    u_0 \\ \vdots \\ u_n
  \end{pmatrix}
  \label{eq:v.Tu}
\end{align}
が成り立つ.
ところで,
\begin{align}
  p_n &= a_{n0} + a_{n1} x + \cdots + a_{nn}x^n
  & &(n=0,1,2,\dots)
\end{align}
とおくと
\begin{align}
  T_{ij}
  &= \frac{\mcL[p_i x^j]}{K_i}
  = \sum_{k=0}^i \frac{\mcL[a_{ik}x^k x^j]}{K_i}
  = \sum_{k=0}^i \frac{a_{ik}\mu_{k+j}}{K_i}
  & &(i, j=0,1,\dots,n)
\end{align}
だから次式が成り立つ.
\begin{align}
  \begin{pmatrix}
    T_{00} & \cdots & T_{0n} \\
    \vdots & \ddots & \vdots \\
    0 & \cdots & T_{nn}
  \end{pmatrix}
  &=
  \begin{pmatrix}
    a_{00}/K_0 & \cdots & 0 \\
    \vdots & \ddots & \vdots \\
    a_{n0}/K_n & \cdots & a_{nn}/K_n
  \end{pmatrix}
  \begin{pmatrix}
    \mu_0 & \cdots & \mu_n \\
    \vdots & \ddots & \vdots \\
    \mu_n & \cdots & \mu_{2n}
  \end{pmatrix}
  \label{eq:T.AH}
\end{align}

\begin{definition}
  \(\functype{\mcL}{\polynoms{\realnums}{x}}{\realnums}\)をモーメント汎関数,
  \(\sequence{\mu_n}{n=0}{\infty}\)を\(\mcL\)のモーメント列,
  \(n\)を非負整数とする.
  このとき,\((n+1)\)次行列
  \begin{align}
    H_n
    &=
    \begin{pmatrix}
      \mu_0 & \mu_1 & \cdots & \mu_n \\
      \mu_1 & \mu_2 & \cdots & \mu_{n+1} \\
      \vdots & \vdots & \ddots & \vdots \\
      \mu_n & \mu_{n+1} & \cdots & \mu_{2n}
    \end{pmatrix}
  \end{align}
  を\(\mcL\)に関する\(n\)次\emph{ハンケル行列}といい,
  \(\Delta_n=\det H_n\)を\(n\)次\emph{ハンケル行列式}という.
  \qed
\end{definition}

\begin{theorem}
  \(\functype{\mcL}{\polynoms{\realnums}{x}}{\realnums}\)をモーメント汎関数とする.
  このとき,\(\mcL\)に関する直交多項式系\(\sequence{p_n}{n=0}{\infty}\)が存在するための必要十分条件は,
  すべての非負整数\(n\)について,\(\mcL\)に関するハンケル行列式\(\Delta_n\)が非零となることである\cite[p. 11, Theorem 3.1]{Chihara2011}.
\end{theorem}
\begin{proof}
  \underline{必要性}.
  直交多項式系\(\sequence{p_n}{n=0}{\infty}\)があれば,
  各\(n=0,1,2,\dots\)について\cref{eq:T.AH}左辺の
  行列が作れて正則になるから,右辺の2つの行列はいずれも正則である.
  したがって\(\Delta_n\neq0\).

  \underline{十分性}.
  多項式列\(\sequence{p_n}{n=0}{\infty}\)を次のように定める.
  まず,\(p_0=1\)とする.
  次に,各整数\(n\:(>0)\)に対してハンケル行列\(H_n\)の第\((n,j)\)小行列式を\(M_{nj}\)とおき,
  多項式\(p_n\)を
  \begin{align}
    p_n &= \sum_{j=0}^n (-1)^{n+j} M_{nj} x^j
    =
    \begin{vmatrix}
      \mu_0 & \mu_1 & \cdots & \mu_n \\
      \vdots & \vdots & \ddots & \vdots \\
      \mu_{n-1} & \mu_n & \cdots & \mu_{2n-1} \\
      1 & x & \cdots & x^n
    \end{vmatrix}
  \end{align}
  で定める.
  \(\Delta_{n-1}\neq0\)だから\(p_n\)は\(n\)次である.
  このとき,\(k=0,1,\dots,n\)について
  \begin{align}
    \mcL[p_n x^k]
    &= \sum_{j=0}^n (-1)^{n+j} M_{nj} \mcL[x^{j+k}]
    =
    \begin{vmatrix}
      \mu_0 & \mu_1 & \cdots & \mu_n \\
      \vdots & \vdots & \ddots & \vdots \\
      \mu_{n-1} & \mu_n & \cdots & \mu_{2n-1} \\
      \mu_{k} & \mu_{k+1} & \cdots & \mu_{k+n}
    \end{vmatrix}
    =
    \begin{cases}
      0 & (k<n) \\
      \Delta_n & (k=n)
    \end{cases}
  \end{align}
  が成り立つ.
  \(\Delta_n\neq0\)だったから\(\sequence{p_n}{n=0}{\infty}\)は直交多項式系となる.
\end{proof}

このやり方だと,当然考察するべき重要な対象について素直に計算してみると重要な定理が出てくるという流れになっていて,個人的に好き.

\printbibliography[title={参考文献}]

\end{document}
